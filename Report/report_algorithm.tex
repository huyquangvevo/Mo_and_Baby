\documentclass[12pt,a4paper]{article}
\usepackage[utf8]{vietnam}
\usepackage{graphicx}
\usepackage{xcolor}
\usepackage{wrapfig}
\usepackage{multicol}
\usepackage{fancyhdr}
\usepackage{fancybox}
\usepackage{hyperref}
\usepackage{amsmath}
\usepackage[left=2.50cm, right=2.50cm, top=2.00cm, bottom=3.00cm]{geometry}
\pagestyle{fancy}
\fancyhf{}
\fancyhead[LE,RO]{\thepage}

\fancyhead[LO]{\small{\itshape{Tìm hiểu về} \textsf{Flex}}}
\begin{document}
\thispagestyle{empty}
\thisfancypage{
\setlength{\fboxsep}{0pt}
\fbox}{}
\begin{center}
\begin{large}
TRƯỜNG ĐẠI HỌC BÁCH KHOA HÀ NỘI
\end{large} \\
\begin{large}
VIỆN CÔNG NGHỆ THÔNG TIN VÀ TRUYỀN THÔNG
\end{large} \\

\textbf{--------------------  *  ---------------------}\\[1.5cm]
\includegraphics[scale=0.35]{12}
\\
\vspace{2cm}
{\fontsize{25pt}{1}\selectfont TÌM HIỂU VỀ \textsf{FLEX}}\\[2cm]
%{\fontsize{15pt}{1}\selectfont Lập trình Web}\\[1cm]
\end{center}
\hspace{3cm}
{\fontsize{12pt}{1}
\selectfont Sinh viên thưc hiện : } \hspace{1pt}
\textbf{\parbox[t]{6cm}{
\selectfont Nguyễn Quang Huy\\[0.3cm]
\selectfont MSSV : 20151690\\[0.3cm]
\selectfont Lớp : KSTN-CNTT-K60\\
   }
}\\[16pt]
\vspace{2cm}
\begin{center}
{\fontsize{12pt}{1}\selectfont HÀ NỘI}\\
{\fontsize{12pt}{1}\selectfont \today}
\end{center}
\newpage
\thispagestyle{empty}
\tableofcontents 
\newpage
\section{Thuật toán Mo}
Cho một dãy số \textit{\textbf{A}} gồm \textit{\textbf{N}} phần tử. Cần thực hiện \textit{\textbf{Q}} truy vấn, mỗi truy vấn \textit{\textbf{(i,j)}} yêu cầu tìm \textit{\textbf{mode($A_i,...A_j$)}}. (Mode của một tâp hợp là giá trị xuất hiện nhiều lần nhất trong tập hợp đó). Giới hạn: \textit{\textbf{N,Q,$A_i\leq10^5$}}.
\subsection{Thuật toán cơ bản}
Có thể giải bài toán bằng thuật toán đơn giản như sau: \par 
\begin{itemize}
\item Với mỗi truy vấn, duyệt từ trái sang phải theo độ dài của câu truy vấn, đếm số lần xuất hiện của các phần tử. 
\item Trong khi đếm thực hiện cập nhật kết quả. 
\end{itemize}
Code đơn giản như sau: \\[0.5cm]
\parbox{500pt}{
\ttfamily
\hspace{0.5cm}void mode(1,r): \par
\hspace{1cm} res = -1; \par 
\hspace{1cm} for i = 1 .. r : \par 
\hspace{1.5cm} count[A[i]] += 1; \par 
\hspace{1.5cm} if res == -1 or count[A[i]] > count[res]: \par 
\hspace{2cm} res = A[i];\par 
\hspace{1cm} return res; \\[0.5cm]
}

Dễ thấy, thuật toán này có độ phức tạp
$\mathcal{O}$\textit{\textbf{(N*Q)}}. Có 2 lí do chính khiến thuật toán này chạy chậm: 
\begin{enumerate}
\item Khởi tạo mảng count mỗi lần mất $\mathcal{O}$\textit{\textbf{(N)}}.
\item Với mỗi truy vấn, phải tính lại mảng count từ đầu. 
\end{enumerate}
Có thể cải tiến thuật toán: \\
Sau khi trả lời truy vấn $\mathbf{[1_1,r_1]}$, để trả lời truy vấn $\mathbf{[l_2,r_2]}$, chỉ cần thay đổi mảng đếm một cách phù hợp. Cụ thể: 
\begin{itemize}
\item Nếu \textit{\textbf{l$_2$ > l$_1$}}, giảm số lần xuất hiện của \textit{\textbf{A$_{l_1}$,...,A$_{{l_2}-1}$}} 
\item Nếu \textit{\textbf{l$_2$ < l$_1$}}, tăng số lần xuất hiện của \textit{\textbf{A$_{l_2}$,...,A$_{{l_1}-1}$}}
\item Tương tự với \textit{\textbf{r$_1$}} và \textit{\textbf{r$_2$}}.
Để cập nhật số lần xuất hiện lớn nhất có thể dùng thêm set. \\
Như vậy, độ phức tạp thuật toán là tổng $|\mathit{l_i - l_{i-1}}| + |r_{i} - r_{i-1}|$, nhân thêm $\mathbf{\mathcal{O}}$\textit{(\textbf{logN})} để đếm và tìm phần tử lớn nhất của mảng đếm.
\end{itemize}
\subsection{Thuật toán Mo}
Thuật toán Mo là một các sắp xếp lại các truy vẫn, sao cho tổng $|l_i - l_{i-1}| + |r_i - r_{i-1}|$ không quá $\mathcal{O}\mathit{(N*\sqrt{N} + Q*\sqrt{N})}$. \par
Thứ tự các truy vấn được định nghĩa qua hàm so sánh dưới đây: \\[0.5cm] 
\parbox{500pt}{
\ttfamily
\hspace{1cm}S = sqrt(N);\par 
\hspace{1cm}{\color{green}bool} {\color{red}cmp}(Query A, Query B)\{ \par 
\hspace{1.5cm} {\color{blue}if}(A.l / S != B.l / S)    	
}
 



\newpage 
\section{Tài liệu tham khảo}
\begin{itemize}
\item \textsf{Lexical Analysis with Flex}
\item \textsf{Flex \& Bison}
\item \href{url}{https://en.wikipedia.org/wiki/Flex\_(lexical\_analyser\_generator)}
\item \href{url}{ftp://ftp.gnu.org/old-gnu/Manuals/flex-2.5.4/html\_mono/flex.html}
\end{itemize}
  
 
\end{document}